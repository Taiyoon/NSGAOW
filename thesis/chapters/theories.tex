\chapter{相关基础知识}\label{chapter:theories}

本章介绍了混合云的基本概念、部署方式以及任务调度研究中使用的隐私加密算法;阐述了用于平衡混合云安全性与效率的多目标优化基本理论知识;并以NSGA-II算法框架为例进行分析,给出元启发式多目标优化算法的原理及其改进方向。为后续章节中隐私任务调度模型的设计与算法优化提供了理论基础。

\section{混合云概述}

云计算作为基于互联网的计算服务模式,通过整合分布式服务器资源向用户提供按需访问的计算能力。这种模式使得计算资源的获取如同水电等公共设施般便捷。云计算的特征是将物理硬件与软件资源抽象为标准化服务,用户仅需关注服务功能需求而无需管理底层基础设施。2006年亚马逊推出的弹性计算云(EC2)标志着云服务的突破性进展,其虚拟化计算资源的租赁机制至今仍是行业标杆。

混合云作为云计算发展的重要方向,通过结合私有云、本地设施和公有云构建出灵活的资源管理模式。还能借助本地就近部署减少网络延迟还可以充分挖掘既有设备的潜力,从而增强系统稳定性并节省运营成本。混合云通过统一平台管理各类资源,当业务需求突然增加时,混合云特有的"云爆发"功能会自动调用公有云资源,及时应对短期的高强度计算需求。这种架构还能满足数据管理法规和企业安全的需求,让隐私敏感的数据始终留在安全环境中。

\subsection{基本概念与术语}

根据NIST标准定义,混合云基础设施由两个或以上独立云实体构成,其中既包含公有云与私有云的协作形式,也支持多个公有云或多个私有云的组合形式。鉴于近年研究与实际应用场景,混合云通常指私有云与公有云的协作形式。本文研究的混合云特指通过任务调度系统实现单公有云与单私有云资源协作的云架构。该架构通过统一的管理平台消除异构环境的差异,既支持私有云资源池的敏感数据处理能力,又可动态调度公有云资源池满足弹性扩展需求,最终在控制成本的同时提高混合云的安全性,并满足隐私数据安全需求。

本文所使用的混合云系统相关技术术语介绍如下:
\begin{itemize}
    \item \textbf{公有资源池}:由第三方服务商托管的多租户共享型云计算资源集合,通过互联网提供标准化服务。其采用弹性架构设计,可根据需求动态调整资源规模,具有按需付费、全局可用性强、运维成本低等特点,适合处理低敏感类数据、波动性业务负载。

    \item \textbf{私有资源池}:组织内部专用的物理或虚拟化资源集群,通过本地数据中心或隔离云环境实现全生命周期管理。其具备物理隔离、定制化配置和细粒度安全管控特征,可确保核心业务数据处理过程的合规性,主要服务于高敏感性或受监管的固定工作负载。

    \item \textbf{混合云资源}:整合公有资源池与私有资源池形成的协同计算架构,通过隐私加密算法实现跨域资源调度。具体而言,采用隐私保护算法对隐私数据进行预处理,将加密后的数据调度至公有资源池的虚拟机执行,同时将核心数据保留在私有资源池处理,以确保安全性\cite{shishidoOptimizingSecurityCost2021, leiHunheYunHuanjing2023}。

    \item \textbf{虚拟机}:基于虚拟化技术实现的逻辑计算单元,可跨异构硬件平台部署独立操作系统环境。在混合云场景中,私有资源池的虚拟机通常采用定制化安全增强配置,而公有资源池的虚拟机则强调快速部署与横向扩展能力,二者通过统一镜像标准实现工作负载的混合编排与跨云处理。
\end{itemize}

\subsection{混合云的部署与应用}

混合云的核心应用价值体现在其“云爆发”功能中。当本地私有云资源接近满载时,系统能够自动将额外的计算任务转移至公有云,这种动态资源分配机制有助于避免服务中断。例如电商促销活动或在线政务服务等具有明显峰谷特征的应用场景,混合云通过预先设定的规则自动扩展公有云资源,既能维持核心业务在私有环境的稳定运行,又可临时调用云端资源处理辅助任务,从而在保障服务质量的同时避免过度硬件投资。

部署混合云需遵循三个基本原则:首先根据数据敏感度对业务进行分类管理,将涉及核心隐私的金融交易等系统固定部署在私有云的专用虚拟机中,而数据分析等非敏感任务则采用公有云资源动态部署。其次建立统一的监控体系,实时评估私有云资源使用状况,当负载指标如处理器或内存使用率或者完工时间达到预设阈值时,启动公有云虚拟机来处理新增负载。最后确保安全策略的跨云同步,在扩展公有云资源时同步实施数据传输加密、访问权限控制等防护措施,形成完整的隐私保护。

该架构在实际应用中展现出多个典型场景:教育机构将核心教学数据保留在私有云,而将视频直播等高峰时段的转码任务分配至公有云虚拟机;医疗信息系统将患者原始数据存储在私有云加密空间,同时将脱敏后的样本上传至公有云进行科研分析;企业将业务系统日常备份至公有云存储,通过定期快照确保应急恢复能力。这些模式共同印证了混合云在安全性和灵活性上的平衡能力。

混合云的部署特性可归纳为三个层面:通过私有云实现敏感数据的物理隔离,满足法规对隐私保护的强制性要求;借助公有云构建弹性混合云资源池,用于处理突发性和非关键任务;最终通过统一调度机制形成跨云协同能力。当私有云负载接近饱和时,系统自动将可公开处理的任务转移至公有云,这种智能调度策略在税务申报、节日促销等周期性高峰场景中展现出显著优势,既保障了核心系统的稳定运行,又降低了基础设施的闲置成本。

\subsection{混合云中隐私加密算法}

在现有混合云任务调度研究中,隐私加密算法主要在用户认证、数据机密性与完整性验证三个方向提供安全保护。任务调度算法根据量化的隐私加密算法安全等级(以0至1区间表示防护强度,1为最高防护),动态选择隐私加密算法,从而提升公有云资源的安全性\cite{wangSecurityawareTaskScheduling2021, leiPrivacySecurityawareWorkflow2022}。

在用户认证方面,系统根据安全需求选择认证方法,如HMAC或AES-CMAC。数据机密性通过对称加密算法实现,例如低安全需求数据采用SEAL等快速的流加密方法以提高效率,中等安全数据采用AES等分组加密算法,而高保密性数据则使用IDEA与混合密钥管理等多层加密方案。完整性检查通过哈希函数实现,例如低安全需求数据采用MD系列计算速度快的哈希算法,而高安全需求数据则采用SHA系列更安全的算法。最后,保留在私有云中独立处理隐私数据的选项,通过私有云的物理隔离为高敏感数据提供最高等级的安全防护。

近年来,随着密码学、可信硬件和分布式计算等新技术的不断发展,一些新型隐私加密算法能够为混合云处理隐私数据提供更高的安全性,但这些技术尚未在任务调度领域得到广泛应用。

基于密码学理论的多方安全计算协议通过分布式协作实现加密状态下的联合运算,适用于跨云数据融合场景,但其复杂计算特性难以满足实时性要求较高的任务。可信执行环境技术利用硬件层的隔离机制,在私有云专属区域保护敏感计算过程,并通过可控的数据流向公有云提升处理效率,但其依赖特定硬件架构限制了部署灵活性。数据脱敏技术通过对敏感数据进行变形处理,在基因分析等统计类任务中展现优势,但存在信息损失带来的计算偏差风险。分布式学习框架如联邦学习,通过本地模型训练与加密参数交互实现全局更新,但在新型攻击面前仍需结合其他加密手段增强防护。然而,这些算法的复杂加密流程、额外计算开销以及对特定数据类型的依赖,限制了其在现有隐私任务调度算法中的有效整合与应用。

\section{多目标优化}

\subsection{多目标优化问题与评价指标}
在现实工程与科学问题中,多目标优化是一个普遍存在的挑战,尤其是在资源分配、调度等场景中,通常需要同时优化多个相互冲突的目标,例如成本、效率与风险。这些目标往往无法通过单一方案同时达到最优,因此传统的单目标优化方法如加权求和法虽然简单,但其依赖于先验的权重设定,难以权衡安全性与效率之间的关系。为了解决这一问题,多目标优化问题通过生成Pareto前沿解集,为决策者提供了多样化的选择空间。以Sun等人\cite{sunEfficientEconomicalEnergysaving2023}在混合云任务调度中的研究为例,一个典型的三目标优化问题可以形式化为以下数学模型:
\begin{equation}
    \begin{aligned}
        & \text{Minimize} \quad f_1(X), \ f_2(Y), \ f_3(Z) \\
        & \text{subject to} \quad g_i(X, Y, Z) \leq 0, \quad i = 1, 2, \dots, m \\
        & \quad \quad \quad \quad h_j(X, Y, Z) = 0, \quad j = 1, 2, \dots, p \\
        & \quad \quad \quad \quad X \in \mathcal{X}, \ Y \in \mathcal{Y}, \ Z \in \mathcal{Z}
    \end{aligned}
\end{equation}
其中,\(f_1(X)\)、\(f_2(Y)\)、\(f_3(Z)\)是需要最小化的三个冲突目标,\(g_i(\cdot) \leq 0\)和\(h_j(\cdot) = 0\)分别表示不等式约束和等式约束,而\(X\)、\(Y\)、\(Z\)为决策变量,分别属于可行域\(\mathcal{X}\)、\(\mathcal{Y}\)、\(\mathcal{Z}\)。

在多目标优化问题中,绝对最优解通常不存在,因此引入Pareto支配关系与Pareto前沿的概念来描述解的优劣。具体而言,对于一个解\(\mathbf{u}\),若其在所有目标上均优于另一个解\(\mathbf{v}\),即满足\(\forall i, f_i(\mathbf{u}) \leq f_i(\mathbf{v})\)且存在至少一个目标\(j\)使得\(f_j(\mathbf{u}) < f_j(\mathbf{v})\),则称\(\mathbf{u}\)支配\(\mathbf{v}\),记为\(\mathbf{u} \prec \mathbf{v}\)。Pareto最优解集则是指所有不被其他解支配的可行解的集合,其对目标空间的映射被称为Pareto前沿。这一理论为多目标优化问题的求解提供了理论基础,强调了在多个冲突目标之间寻找平衡解的必要性。

为了量化多目标优化算法的性能,常用的评价指标包括世代距离(Inverted Generational Distance, IGD)和超体积(Hypervolume, HV)。这两个指标在多目标优化中具有重要的评价意义,常用于衡量算法在收敛性与解集质量方面的表现。

反转世代距离(Inverted Generational Distance, IGD)是一种常用的评价指标,其通过计算真实Pareto最优解集到非支配解集的平均距离,提供了一个对算法收敛性与解集质量的综合评估。,其计算公式为:
\begin{equation}
    \text{IGD}(\mathcal{P}, \mathcal{P}^*) = \frac{\sum_{i=1}^{F^*} d(v_i, \mathcal{P})}{F^*}
\end{equation}
其中,\(\mathcal{P}\)为算法解集,\(\mathcal{P}^*\)为真实Pareto前沿,\(F^*\)为真实Pareto前沿中解的个数,\(v_i\)为真实Pareto前沿中的第\(i\)个解,\(d(v_i, \mathcal{P})\)表示真实Pareto前沿中第\(i\)个解到算法解集\(\mathcal{P}\)的最近欧氏距离。IGD指标通过计算真实Pareto前沿中每个解到算法解集的平均最短距离,能够更全面地评估解集的收敛性和多样性。

超体积(HV)则是通过计算非支配解集与参考点在目标空间中所围成的超立方体体积来评估解集的质量,对于三目标优化问题其定义如下:
\begin{equation}
    \text{HV}(\mathcal{P}, r) = \lambda_3 \left( \bigcup_{x\in \mathcal{P}} \prod_{i=1}^{3} [f_i(x), r_i] \right)
\end{equation}
其中,\(\lambda_3\)表示三维空间中的勒贝格测度,即体积,\(f_i(x)\)表示第\(i\)个解在第\(i\)个目标上的归一化值,\(r\)为参考点。HV值越大,表明解集的覆盖范围与分布均匀性越好。虽然HV指标能够综合反映解集的收敛性与多样性,但其计算复杂度较高,且参考点的选择对结果准确性有一定影响。

世代距离(GD)也可用于衡量算法解集与真实Pareto前沿之间的逼近程度,通过计算算法解集与真实Pareto前沿解集之间的最小欧氏距离来衡量解的收敛性,其值越小表明解的收敛性能越优。GD与IGD对真实Pareto最优解集的依赖性较强,若选取的最优解集不准确,可能会导致对算法性能的错误评估。

解决多目标优化问题的主要方法可分为精确算法与元启发式算法两类。精确算法基于严格的数学模型,如分支定界法试图求得理论最优解,但其计算复杂度往往随问题规模呈指数级增长,仅能有效处理小规模或特殊结构的优化问题。而实际工程、生物信息学、运筹学等领域的优化问题通常具有高维度、非线性、强约束等复杂特性,且大多数被归类为NP-hard问题,除非NP等于P,否则无法在多项式时间内精确求解。因此,研究者转向近似方法,即元启发式算法,以在合理时间内找到近似最优解。

元启发式算法基于自然界或数学规律抽象出的搜索策略,主要分为两类:一类受现实世界启发(如粒子群优化模拟鸟群觅食、遗传算法模拟生物进化、模拟退火模拟冶金退火过程),另一类源于数学或抽象规则,例如禁忌搜索通过禁忌表避免搜索重复、可变邻域搜索通过动态切换邻域结构平衡探索与开发。这些算法通过在解空间中高效搜索权衡解,能够在复杂场景下逼近多目标Pareto前沿,尤其适用于大规模离散优化与动态优化问题\cite{abdel-bassetChapter10Metaheuristic2018}。这种通用性使得元启发式算法成为解决现代复杂多目标优化问题的核心工具。下一小节将介绍多目标元启发式算法的编码类型与离散优化编码方式,并以NSGA-II为例分析其框架设计,最后探讨算法的发展方向。

\subsection{多目标元启发式算法}

多目标优化领域的元启发式算法主要分为群智能算法与进化算法两类。群智能算法如多目标粒子群优化(MOPSO)和多目标灰狼优化(MOGWO),基于群体协作行为求解问题;进化算法则以多目标遗传算法(如NSGA-II、MOEA/D、SPEA-II)为代表,通过模拟生物进化机制实现全局搜索。其中,NSGA-II\cite{debFastElitistMultiobjective2002}采用快速非支配排序与拥挤距离度量解质量,并通过精英选择策略维持种群进化方向,其时间复杂度从$O(MN^3)$降低至$O(MN^2)$,显著提升了大规模问题的求解效率。在4目标以内优化问题上,NSGA-II展现出逼近真实Pareto前沿的能力。本小节介绍了多目标元启发式算法支持的实数、离散与二进制等多种编码类型,并介绍离散优化问题在元启发式算法中的编码方式。并以NSGA-II为例,详细介绍了多目标遗传算法框架中适用于离散编码的经典遗传算子与精英选择策略设计,为后续引入改进措施提供了理论基础。最后,介绍多目标元启发式算法的发展方向。

多目标元启发式算法的编码方案主要分为三类:实数编码、整数编码和二进制编码。其中,实数编码作为一种通用方案,可通过编解码方式灵活表示多种优化问题,但其基于连续搜索空间的特性易忽略离散约束,导致局部最优问题\cite{abdel-bassetBinaryMetaheuristicAlgorithms2024};整数编码专门适配离散优化问题,例如本文研究的虚拟机分配编码,但其需定制交叉和变异算子以保证解的合法性并提高优化速度;二进制编码则通过离散变量的直接映射,例如本文中的执行模式选择,支持特定问题建模,但在大规模场景下易因编码冗余降低效率。相较基于位置更新机制的群智能算法,例如PSO的粒子速度模型,NSGA-II在离散编码的优化上更具灵活性,其染色体结构更易实现离散空间搜索。以组合优化为例,NSGA-II通过可以离散编码表达任务序列,并设计交叉算子避免非法解产生,可以适应不同的编码方案。

任务调度等问题属于离散优化问题,其编码方法主要分为两类:第一类沿用实数编码方案,通过 Sigmoid 等传递函数将连续值映射为离散决策变量\cite{panSurveyBinaryMetaheuristic2023}。该方法保留了传统元启发式算法连续编码的结构特性,但存在编码效率低、难以精准描述离散目标函数特征的问题,且连续搜索区间易导致算法陷入局部最优;第二类则直接采用离散编码方案,虽通过减少决策空间加速收敛,但其固有的离散特性易导致个体状态信息丢失,且只能使用离散交叉与变异算子,从而影响搜索质量。为适应任务调度等离散优化问题的特殊需求,需结合定制化遗传算子以保持解的合法性与多样性,从而提高算法的搜索速度与质量。

多目标遗传算法的遗传算子主要包括变异与交叉算子,二者在遗传算法中发挥着不同的作用。变异算子通过引入新的编码方案,能够增加种群的多样性,从而增强全局搜索能力。然而,变异操作也可能破坏已有的优质编码,特别是在变异概率过高的情况下,遗传算法可能会退化为随机搜索,进而延长收敛时间。位翻转变异是一种典型的变异算子,它通过随机选择个体基因的某一位并将其值进行翻转,从而生成新的编码方案。然而,固定概率的位翻转可能导致收敛速度缓慢,因此可以通过动态调整变异概率来优化表现。例如,基于位翻转的非均匀变异算子能够结合进化代数和基因位状态进行变异概率的动态调整,在初期以较高的概率引入多样性以增强全局探索能力,而在后期逐步降低变异强度,从而加速向高质量解的收敛。

交叉算子则主要负责在个体之间交换优秀的部分编码方案,以提高局部寻优能力。然而,交叉操作所能生成的编码范围相对有限,可能导致种群多样性下降,进而影响非支配解集的质量。多点交叉是一种广泛应用于多种编码类型的交叉算子,它通过在多个交叉点对父代基因进行分块重组,适用于解空间基因无明显位置依赖性且存在多峰局部最优的优化场景。然而,由于多点交叉随机选择交叉点,可能导致具有关联关系的部分编码难以同时被转移。因此,一些研究采用两点交叉代替多点交叉,这种方式既能充分交换编码信息,又能保留部分关联关系,从而在提升解集质量的同时维持种群的多样性。

针对离散优化问题的特性,NSGA-II的精英选择策略通过非支配排序与拥挤距离(Crowding Distance)机制共同实现解的筛选与种群多样性维护。首先,算法对合并种群执行快速非支配排序,将种群划分为不同Pareto非支配等级的子集,按优先级顺序依次为 \(F_1, F_2, F_3, \dots, F_l\);随后,按照等级顺序将各子集依次保留并存入下一代种群中,直至种群规模超过设定值 \(N\);最后,对存入种群的最后一等级子集,通过拥挤距离机制顺次择优保留,直至种群规模达到 \(N\)。拥挤距离通过以下公式计算:
\begin{equation}
    \text{Crowd}(s) = \sum_i \text{norm}(f_i(s))
    \label{eq:nsga2-crowd}
\end{equation}

拥挤距离作为衡量解分布稀疏性的关键指标,用于维持解集的多样性和优化算法的搜索能力。当非支配解数量超出种群容量时,系统基于拥挤距离值进行筛选,通过优先保留拥挤距离较大的解,自动淘汰分布密集区域的个体,从而确保解集维持良好的多样性分布,提升算法在多目标优化问题中的全局搜索性能。

最后,本节介绍多目标元启发式优化算法的未来发展方向。首先,优化超过三个目标函数的超多目标优化问题(Many Objective Optimization)面临挑战。随着目标维度的增加,传统基于几何距离的NSGA-II拥挤度机制难以准确反映个体在高维空间中的实际距离,导致高维空间中解集的分布性与多样性,进而影响调度结果。未来研究可通过融合参考点法(如NSGA-III)或指标选择策略(如IBEA的指标选择\cite{zhengIndicatorBasedMultiobjective2024}),提升优化质量。其次,根据无免费午餐定理,没有一种通用算法能在所有问题上表现最佳,针对具体场景专门设计的优化算法可以提高效率和结果质量。例如,在本文的混合云隐私任务调度中,通过改进遗传算子提升任务调度质量并加快收敛速度。最后,针对不同操作步骤的特点,结合问题需求和算法运行阶段动态调整参数,能够提升算法性能。例如,在算法初期,通过增加变异概率来扩大搜索范围并保持多样性;在算法后期,降低变异概率并提高交叉概率,有助于更快找到更精确的解。

\section{本章小结}

本章介绍了多目标优化、混合云架构及其隐私加密算法的基础知识。首先,阐述了混合云的基本概念、术语和部署模式,并介绍了混合云任务调度中常用的隐私加密算法,这些算法能在用户认证、数据保密性和完整性检查三个方面提供安全保护。其次,介绍了多目标优化问题的核心概念、评价指标及使用的元启发式算法。重点分析了NSGA-II的遗传算子在离散优化问题中的应用,并介绍了算法未来发展方向。
