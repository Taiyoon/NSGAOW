\chapter{总结与展望}\label{chapter:conclusion}

\section{总结}

混合云架构结合了私有云的隐私安全能力与公有云的丰富资源,实现了对隐私数据的高效处理。随着数据隐私安全法规的不断完善以及隐私保护需求的日益提升,如何在调度隐私任务时平衡安全性与效率成为关键挑战。针对此挑战,本文构建了动态细粒度的隐私任务调度模型,并设计了多目标隐私任务调度算法,以优化完工时间、安全性与成本三个目标,最后输出一组在完工时间(效率)、安全性与成本上满足非支配条件的调度方案集合,使用户可以选择安全和效率间的平衡点。

本文工作总结如下:

\begin{enumerate}
    \item 分析了混合云隐私任务调度领域的研究现状,发现现有调度方法在效率与安全性权衡方面存在不足,同时在细粒度隐私保护、公私云间协作方面需要进一步加强。基于此,本文制定了建立动态细粒度的隐私任务调度模型并设计多目标隐私任务调度算法的研究方案。
    \item 构建了动态细粒度的隐私任务调度模型。通过公私云协作的动态任务模型,精确建模跨云协作中隐私加密算法的开销,为优化公私云协作效率打下基础。通过细粒度的隐私数据标签,对隐私数据进行精准保护,并提出混合云安全性指标,量化低安全系数隐私加密算法对隐私数据的潜在风险,从而提升混合云安全性,并为平衡安全性与效率提供基础。
    \item 设计了NSGA-OW多目标隐私任务调度算法,结合虚拟机空闲时段优化算法并改进了遗传算子,同时优化完工时间、安全性与成本指标,实现了效率与安全性的权衡。
    \item 通过实验验证了模型与算法的有效性。对实验结果的分析表明,由于本文考虑了安全性与效率的平衡、细粒度的隐私标签以及公私有云间精确的协同,相较于传统的单目标隐私调度算法,本文方法在完工时间、安全性与成本方面具有一定优势。同时,本文设计的NSGA-OW算法相比传统的多目标元启发式算法,在收敛速度与非支配解集质量上表现出更好的性能,验证了模型与算法的有效性。
\end{enumerate}

\section{展望未来}

混合云作为隐私数据处理的理想平台,在企业数字化转型与政务处理等领域具有重要价值。虽然本文提出的动态细粒度的隐私任务调度模型与多目标调度算法解决了混合云中效率与安全的权衡等问题,但由于作者水平与时间限制,仍有一些问题需要未来进一步研究:

\begin{enumerate}
    \item 考虑具有复杂依赖关系的任务场景。本文虽然分析了公私云协同处理时的任务内部依赖关系,但所提出的调度模型仍只适用于任务间无依赖的简单场景。对于涉及任务间依赖关系的科学工作流,或包含分支与循环结构的复杂隐私保护算法,本文模型存在局限性。未来可以研究可动态调整拓扑结构的DAG工作流\cite{houssam-eddineHPCDAGTaskModel2021}等方法,以应对这些复杂场景。
    \item 结合数据动态放置与任务调度,进行联合优化。为了简化模型,本文假设隐私数据存储于指定的私有虚拟机中,这难以应对热点数据引发的负载不均衡问题。后续研究可结合数据动态放置与缓存策略\cite{kchaouPSOTaskScheduling2022,wangCostDrivenDataCaching2021},优化数据存储位置与访问效率,实现私有虚拟机负载均衡,从而提升混合云处理热点数据的能力。
    \item 进一步提高算法的实时性以增强其实用价值。尽管NSGA-OW算法通过改进遗传算子等技巧提升了收敛速度,但其多目标遗传算法的本质使得优化速度仍低于启发式算法。未来研究可对优化问题进行分析与分解,将部分优化目标转化为近似凸目标以加速求解,此外,还可结合机器学习或启发式等方法进一步提升算法运行速度。
\end{enumerate}
