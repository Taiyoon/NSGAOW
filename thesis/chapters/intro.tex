\chapter{绪论}\label{chapter:intro}

\section{研究背景及意义}

近年来,随着数字经济的迅猛发展,云计算作为关键信息基础设施,在政府治理与企业数字化转型中发挥着核心作用。根据中国信息通信研究院发布的《云计算蓝皮书(2024年)》,全球云计算市场正以18.6\%的年复合增长率快速扩张,预计到2027年市场规模将突破万亿美元\cite{ZhongGuoXinXiTongXinYanJiuYuanYunjisuanlanpishu2024nian2024}。在这一背景下,混合云作为云计算的重要模式,通过整合私有云的安全可控性与公有云的弹性扩展优势,成为具有隐私数据处理需求企业数字化转型的重要选择。企业可将隐私数据部署于私有云以确保完全控制权,同时利用公有云的弹性算力进行临时扩容,既满足隐私安全要求,又能快速响应市场需求。随着企业对混合云隐私数据处理需求的持续增长,华为云、阿里云、亚马逊AWS等主要厂商纷纷推出成熟的混合云解决方案,进一步推动了混合云技术的普及与应用。总之,混合云凭借其架构中独特的可控性和灵活性等优势,逐渐成为处理隐私数据的理想平台。

尽管混合云在隐私数据处理上具备灵活性和可控性优势,但混合云中的隐私安全依旧是制约大规模推广部署的重要挑战。首先,混合云需管理多个云平台,其复杂的资源组成导致攻击面显著扩大,数据泄露事件频发,例如酒店消费信息和订票网站信息频繁遭窃取,表明混合云环境中的部分资源存在安全风险;其次,混合云必须满足严格的合规性要求,在隐私数据传输过程中采用符合中国《个人信息保护法》或欧盟GDPR等法规要求的加密技术,并针对不同监管需求动态选择适配的加密策略;最后,特定行业的隐私数据需要强化保护措施,例如医疗行业在将患者记录迁移至云平台时,必须满足HIPAA规范,以确保患者健康信息的隐私与安全得到有效保障。为应对这些挑战,现有混合云任务调度研究通过引入隐私加密算法,围绕数据传输保密性、用户认证以及数据完整性三个方面构建安全机制,从而显著提升混合云的隐私保护性能\cite{leiPrivacySecurityawareWorkflow2022, wangSecurityawareTaskScheduling2021, chenSchedulingWorkflowsSecuritySensitive2017, budatiSecureMultiLevelPrivacyProtection2023, sonPrivacyProtectionBased2020, balJointResourceAllocation2022}。除此之外,存在一些尚未在混合云任务调度中被充分利用的新型隐私加密算法。例如,同态加密适用于投票系统中的密文计算,数据脱敏技术在大数据分析场景中表现突出,而局部模糊处理则可以有效遮挡人物图像、车牌路牌等敏感信息。这些算法虽然能够提供更强的安全机制,然而,这些新型隐私加密算法存在复杂加密流程、额外计算开销以及仅适用于特定类型隐私数据的局限性,使得现有隐私任务调度算法难以有效整合和应用这些技术。综上所述,通过在混合云任务调度中引入隐私加密机制,可以部分解决混合云部署中遇到的隐私安全问题,提高混合云的部署率。
% 然而,存在什么问题?
% 然而,当前混合云隐私任务调度中,难以权衡安全性和效率,对于隐私数据的安全需求和隐私数据的具体类型缺乏感知,且公私云效率不佳,导致新型隐私加密算法难以部署。

最后,混合云在隐私数据处理领域展现出的独特潜力使其成为全球数字化转型进程中的重要支撑力量,各国政府与企业不仅通过政策框架加速技术落地,更将其纳入国家数字化发展战略的核心议程。美国发布《国家网络安全战略》,强化云基础设施安全与弹性建设;欧盟在《欧洲2030》战略规划中提出75\%的企业需采用云计算、大数据与人工智能技术,以提升数字竞争力;中国印发《算力基础设施高质量发展行动计划》,推动云服务模式整合算力资源,实现多元算力高效协同。随着隐私数据处理需求的持续增长,混合云环境下的隐私任务调度研究成为提升数据处理效率、保障数据安全的关键课题,其重要性日益凸显。

\section{国内外研究现状}\label{sec:related-works}

本章从隐私任务调度中保护隐私数据的方法,以及云计算中单目标和多目标任务调度的研究进展两个方面,对现有文献进行系统性梳理与对比分析,从而明确混合云中隐私任务调度研究的核心问题与发展方向。

\subsection{隐私任务调度研究现状}

隐私任务调度是隐私数据处理中的关键问题,现有隐私任务调度研究主要通过资源限制、安全评估与安全加固三种方法保护隐私数据安全\cite{hammoutiSecuredWorkflowScheduling2023}:资源限制方法通过限制敏感数据只使用可靠的计算资源处理,而将非敏感数据分配到所有计算资源;安全评估则根据隐私数据的风险级别和计算资源的安全评级,动态选择计算资源;安全加固机制利用隐私加密算法增强计算资源可靠性,允许任务使用更多的计算资源。这三种方法之间存在着从“静态隔离”到“动态评估”再到“主动加固”的递进关系。

在资源限制方法研究方向,当前研究通过设定任务执行位置约束实现隐私数据的保护。\cite{huangImprovedGeneticAlgorithm2023, sunEfficientEconomicalEnergysaving2023, leiPrivacySecurityawareWorkflow2022, wangSecurityawareTaskScheduling2021, sharifPrivacyAwareSchedulingSaaS2017, stavrinidesDynamicSchedulingBagsoftasks2021, wenSchedulingWorkflowsPrivacy2020, ZhangMinYuYigouhunheyunhuanjingxiadaiyinsixingyueshudeSparkrenwutiaodu2023,huangImprovedGeneticAlgorithm2023, budatiSecureMultiLevelPrivacyProtection2023},通过对任务施加资源限制,确保隐私任务仅使用安全的计算资源执行,避免隐私数据泄露。

Stavrinides等人\cite{stavrinidesDynamicSchedulingBagsoftasks2021}构建的混合云隐私调度框架,将包含敏感数据的任务袋分配至私有云资源,考虑截止时间和安全约束的任务调度问题,优化资源分配可以节约高成本并确保数据安全。
Sharif等人\cite{sharifPrivacyAwareSchedulingSaaS2017}研究了隐私工作流调度问题,重点关注资源限制策略以实现隐私保护。该研究以医疗环境中的隐私任务调度为例,通过将含有患者身份信息的任务分配至私有云执行,确保了数据安全性。同时,利用公有云资源处理非敏感任务,在保障隐私的基础上有效降低了成本并兼顾了任务截止时间约束。
张敏禹\cite{ZhangMinYuYigouhunheyunhuanjingxiadaiyinsixingyueshudeSparkrenwutiaodu2023}针对Spark工作流任务提出的两阶段协同框架,通过引入弹性资源抢占机制与动态时间窗松弛技术,在保障隐私数据在私有云处理的前提下,优化了截止期达成率并提高了资源消耗均衡性。

资源限制方法虽通过执行位置约束实现了隐私保护,但其也存在计算资源利用不充分的不足。制约了不同计算环境之间的协同能力,难以适应复杂多变的负载与安全态势。Stavrinides\cite{stavrinidesDynamicSchedulingBagsoftasks2021}的研究表明,当任务中隐私任务的比例从25\%上升至75\%时,截止时间违反率提升了37\%,反映了资源限制策略在动态环境下的局限性。
为了更充分地利用计算资源,另一些研究从安全评估的角度出发,通过动态评估计算资源的可信度以及隐私数据的安全需求\cite{zhouPrivacyRegulationAware2019, YuYiHanJiyuduocengmohuzonghepinggudeyinsibaohuxiaoguopinggufangfa2020, asghariPrivacyawareCloudService2022, singhSchedulingRealTimeSecurity2021, dengYidongBianyuanJisuan2023, zhuTaskSchedulingMultiCloud2021},根据数据安全级别的高低,合理分配至不同可信度的计算资源,以此实现计算资源的最优配置。

Asghari等人\cite{asghariPrivacyawareCloudService2022}构建的云服务组合方法通过多维度服务分级与信任评级体系,实现隐私保护级别与云服务商安全能力的自动化匹配,提升了整体服务质量。
针对跨国场景中的合规性挑战和多级数据保护要求,Zhou等人\cite{zhouPrivacyRegulationAware2019}设计一种进程映射方法,通过动态约束条件的组合优化算法,在任务调度和数据跨境传输符合隐私保护法规与传输时延控制,通过数学建模和启发式算法设计,优化了通信成本和部署性能。
在多方数据融合场景领域,刘圣龙等人\cite{LiuShengLongMianxiangduofangshujuronghefenxideyinsijisuanjishuzongshu2024}系统梳理了多方协同计算架构下的隐私保护技术演进路径,提出覆盖数据全生命周期的安全技术,其分析表明后续研究应着重解决跨域信任建立、计算-通信开销优化等关键问题。
邓慧娜\cite{dengCheLianwangBianyuan2023}提出了一种用于车联网环境的轨迹隐私保护方法。该方法通过动态调整任务分配的方案,同时考虑迁移成本,不仅能够快速提供所需服务,还构建了一套针对边缘计算环境中可能存在攻击的多层次隐私保护评估标准。

现有研究通过多种方法评估计算资源的可信度和隐私数据的安全需求,以优化资源的分配利用。然而,这些研究通常局限于特定应用场景,也无法满足用户对数据的细粒度隐私保护需求\cite{liSurveyPrivacypreservingOffloading2022},例如不同地区的隐私法规差异,导致隐私安全性下降,且同类计算任务需要重复设计专用的隐私处理方案。因此,可以引入一种细粒度的隐私数据标签机制,以灵活满足不同用户的多样化隐私保护需求。

上文介绍的资源限制与安全评估两种方法主要通过限制数据处理位置的方式来安全处理隐私数据,存在计算资源利用不充分的不足。而安全加固机制则通过引入隐私加密算法,提升低可信度计算资源的安全性\cite{leiHunheYunHuanjing2023, wangSecurityawareTaskScheduling2021, chenSchedulingWorkflowsSecuritySensitive2017, budatiSecureMultiLevelPrivacyProtection2023, sonPrivacyProtectionBased2020, balJointResourceAllocation2022},从而使其能够处理部分隐私数据。

雷剑\cite{leiHunheYunHuanjing2023} 提出隐私工作流增强框架采用分级加密策略,依据任务敏感度动态选择不同等级的加密算法,增强了隐私工作流在混合云中传输数据的安全性,在保持数据隐私性的前提下提升混合云整体资源利用效率。
Li等人\cite{liSecurityPerformanceawareResource2020}针对当前移动边缘计算领域的研究多聚焦于任务卸载与性能优化,却普遍忽视企业多媒体在无线传输中的安全风险的问题。通过Lyapunov优化模型动态调配加密策略与计算资源,在保证安全性的前提下将移动边缘计算环境下降低传输时延与能耗。
Chen等人\cite{chenSchedulingWorkflowsSecuritySensitive2017}提出安全敏感中间数据的工作流调度框架,通过预计算资源空闲时段进行任务复制减少数据传输与加密对后续任务启动时间的影响,实现实现安全约束下完工时间与成本的联合优化。
Hammouti等人\cite{hammoutiWorkflowSecurityScheduling2021}提出了一种针对混合云环境的任务调度策略,通过引入数据加密与认证机制增强混合云资源的安全性和经济性,重点研究了这些安全服务对任务总成本和截止时间的影响。

这些考虑安全加固研究通过在任务调度中引入隐私加密算法,提升数据传输的保密性、完善用户认证机制并确保数据完整性,从而使部分低可信度的计算资源也能够安全地处理隐私数据。
然而,这些研究忽视了隐私加密算法会为系统带来额外的加解密开销,占用更多的网络与计算资源。当前的研究大多将隐私加密算法简单地视为任务处理过程中引入的一个固定延迟\cite{leiHunheYunHuanjing2023, hammoutiWorkflowSecurityScheduling2021},缺乏对其复杂影响的深入分析。实际上,加密和解密过程会引入额外的任务依赖关系,例如,在混合云环境下处理隐私数据时,公有云的处理任务依赖于私有云的加密任务,而私有云的验证任务又依赖于公有云的处理任务。现有安全加固任务调度研究未充分考虑加密与解密引入的额外任务依赖关系,导致安全加固方法的任务处理效率降低。

% 针对现有安全加固的任务调度研究未充分考虑隐私加密算法开销的问题,本小节分析了调度期间会动态变化的任务调度研究。在混合云调度隐私任务时,系统会根据预期完工时间等外部环境信息决定是否需要增加隐私加密算法作为子任务,以提高调度质量。

针对现有安全加固任务调度研究未充分考虑隐私加密算法开销的问题,本小节分析了调度期间支持动态变化的任务调度研究。Stavrinides等人研究了分布式系统中具有动态变化结构的线性工作流\cite{stavrinidesSchedulingLinearWorkflows2023, stavrinidesMulticriteriaSchedulingLinear2021}(Linear Workflow, LW)及其安全感知调度技术\cite{stavrinidesSecurityAwareOrchestrationLinear2022}。LW在调度过程中允许动态的添加与删除子任务,以更好地适应外部环境。并提出了一种基于条件允许部分计算的调度方案,针对高风险任务需要安全资源处理、低风险任务可灵活分配的场景,设计了两种路由技术。最后,通过仿真实验验证了不同安全资源比例、任务风险概率以及子任务的动态插入与删除对性能的影响。然而,他们的研究未能将线性工作流应用于隐私加密算法的场景,也未能通过动态插入加密与验证任务以适应混合云环境下的隐私任务调度需求。

% 本小节通过对隐私任务调度相关研究的梳理,本文发现当前隐私任务调度研究在满足数据细粒度隐私需求、对隐私加密算法算法的精确建模以及安全性量化评估等方面存在不足。考虑安全评估的研究\cite{sonPrivacyProtectionBased2020, zhouPrivacyRegulationAware2019}未能有效考虑用户对数据的细粒度隐私需求,导致隐私安全性下降,且同类计算任务需要重复设计专用的隐私处理方案。此外,考虑安全加固的任务调度研究未充分考虑隐私加密算法会引入额外任务依赖关系的问题\cite{caiFailureresilientDAGTask2021},从而导致该调度方法在实际中任务执行效率下降。为解决这一问题,本文分析在调度过程中支持任务拓扑结构的动态调整的研究\cite{stavrinidesSchedulingLinearWorkflows2023},以准确建模隐私加密算法开销。最后,现有研究大多将隐私安全作为约束条件,而缺少对安全性的量化评估,难以在安全与效率之间实现有效权衡\cite{leiPrivacySecurityawareWorkflow2022, shishidoOptimizingSecurityCost2021, wangSecurityawareTaskScheduling2021}。

本小节通过对隐私任务调度相关研究的梳理,总结了当前研究的不足。首先,安全评估的任务调度研究\cite{sonPrivacyProtectionBased2020, zhouPrivacyRegulationAware2019}未能充分考虑数据所有者对数据的细粒度隐私需求,导致隐私安全性降低,且需要为同类任务重复设计专用的隐私处理任务。其次,安全加固的任务调度研究\cite{caiFailureresilientDAGTask2021}忽略隐私加密算法引入的额外任务依赖关系,降低了任务执行效率。为此,本文参考支持任务拓扑结构动态调整的研究\cite{stavrinidesSchedulingLinearWorkflows2023},在混合云隐私任务调度中,系统会根据预期完工时间等信息,决定是否增加隐私加密算法作为子任务,从而提升调度质量。最后,上述三类研究多将隐私安全作为约束条件,缺乏对安全性的量化评估,难以在安全与效率之间实现有效权衡\cite{leiPrivacySecurityawareWorkflow2022, shishidoOptimizingSecurityCost2021, wangSecurityawareTaskScheduling2021}。为了权衡安全性与效率,下一小节会分析云计算中多目标与单目标的任务调度研究现状,以找到适合混合云隐私任务调度研究的调度方法。

\subsection{云计算任务调度现状}

本小节从单目标优化与多目标优化两种云计算中任务调度问题建模方法分析云计算任务调度现状。
单目标优化作为云计算任务调度的传统方法,通常以完工时间、价格、能耗等用户最关心的指标为优化目标,并将资源、截止时间及安全性等次要指标作为约束条件进行权衡,从而兼顾系统多方面性能需求\cite{abrahamMultiObjectiveOptimizationTechniques2025}。

在混合云隐私任务调度中,许多研究使用了单目标优化方法,通过资源限制约束,只允许私有云处理包含隐私数据的任务,例如Wang等人\cite{wangSecurityawareTaskScheduling2021}设计了将安全约束与最早截止时间优先调度结合的任务调度策略,平衡了截止时间违反率和安全性,并通过实验验证这种方法降低了了截止时间。
而在边缘计算中,张智峰\cite{ZhangZhiFengBianyuanbianyuanxiezuohuanjingxiarenwutiaodufangfayanjiu2023}利用软件定义网络实现任务卸载与资源动态调配,并通过资源约束与执行位置约束兼顾了边-端服务器的负载均衡,降低了任务完成时间,并提高了边缘服务器的负载均衡性。

单目标优化方法通过优化首要目标并将次要目标作为约束,简化了任务调度问题,且在多数场景下取得了良好效果,但在混合云隐私任务调度这类优化目标相互冲突的场景中存在局限。例如,由私有云独立处理任务虽然安全性高但资源受限,而公私云协作处理虽然高效却可能泄露隐私。

为了优化相互冲突的目标,还有研究考虑云计算中的多目标优化问题。云计算中的多目标优化问题是指同时优化两个或多个冲突的目标,并在成本、性能、能效和用户满意度等目标之间取得平衡。Khan的综述\cite{khanReviewTaskScheduling2023}指出,云计算任务调度研究已涵盖20类调度指标,但与隐私安全相关的新兴指标尚未得到充分研究。目前,多目标优化问题有两种解决思路,一种是通过加权求和将多目标问题转化为单目标问题,再使用单目标调度算法求解;另一种是使用多目标调度算法,生成一组非支配的调度方案集合,非支配的调度方案是指无法在改进一个目标的同时不损害其他目标。

Laili等人\cite{lailiParallelSchedulingLargeScale2023},针对工业物联网场景下大规模任务的云-边缘协作调度问题,提出了一种基于加权求和的多目标优化方法,通过将任务完成时间与能源消耗合并为单一目标函数实现高效调度。该方法结合并行分组-合并进化算法,提升了任务分配的均衡性与计算效率。
如Sun等人\cite{sunEfficientEconomicalEnergysaving2023}针对混合云工作流调度问题,构建总延迟-私有云能耗-公有云成本的三目标优化模型,提出基于鲈鱼群算法的双阶段优化策略,通过探索Pareto前沿解集与贪心搜索的协同机制,在仿真实验中实现能耗与经济成本的均衡下降。
\cite{wangIntegratingWeightAssignment2018}NSGA-II非支配排序多目标优化能力的基础上,通过定性指标确定搜索方向,加快了优化算法在用户感兴趣的目标上的搜索速度。
Mousavi等人\cite{mousaviDirectedSearchNew2023}基于云计算任务调度算子,论文在经典NSGA-II算法中引入了一个新的重组算子,从而提出了一种定向非支配排序遗传算法(D-NSGA-II)。该新算子能够调节种群的选择压力,有效平衡算法在全局探索与局部开发之间的能力。
Mangalampalli等人\cite{mangalampalliMultiObjectivePrioritized2024}提出一种基于深度强化学习的多目标优化调度框架,通过优先级感知与DQN动态决策机制显著降低科学工作流调度场景的完工时间与能源消耗。

% 根据以上研究可以发现,使用加权求和方法的多目标优化算法将多目标问题转化为单目标优化问题,但其调度效果高度依赖于加权系数的设定,而加权系数往往难以准确确定,难以适应本文考虑的隐私与效率难以权衡的场景。而基于元启发式算法的多目标优化方法通过模拟生物进化过程中的选择、交叉与突变机制,能够同时优化多个目标,并为用户提供一组可比较的非支配调度方案,从而在隐私任务调度场景中展现出更强的适应性和优势。

从以上多目标任务调度研究中可以看出,加权求和法将多目标问题转化为单目标优化,但调度效果依赖加权系数,而这些系数难以准确设定,难以应对安全与效率这一对目标的权衡问题。而基于元启发式算法的多目标优化方法通过模拟生物进化机制(如选择、交叉与突变),能够同时优化多个目标,并为用户提供一组非支配调度方案,在需要平衡安全与效率的隐私任务调度场景中更有优势。

本小节总结了混合云中单目标和多目标任务调度的研究现状,发现单目标调度在平衡相互冲突优化目标方面存在不足,基于加权求和的多目标优化算法又难以确定加权系数,而多目标元启发式优化算法不仅能优化相互冲突的目标,还能避开加权系数的设定问题,提供质量更高的调度方案。因此,本文选择元启发式算法的多目标优化方法调度混合云中的隐私任务,以有效权衡效率与安全。

\section{论文研究内容和主要工作}

本文针对现有混合云隐私任务调度研究中安全性与效率难以权衡的问题,建立混合云中动态细粒度的隐私任务调度模型,并以最小化完工时间、成本并最大化系统安全性为目标建立多目标优化问题。针对该优化问题,设计多目标隐私任务调度算法,能同时优化三个目标,并为用户提供一组非支配调度方案集合。最后,通过实验验证模型与算法的有效性。
本文的主要研究如下:

\begin{enumerate}
    \item 提出一种混合云动态细粒度隐私任务调度模型,以解决隐私任务调度中的效率与安全平衡难题,同时为调度算法提升调度的安全性与效率提供基础。该模型通过线性工作流技术构建公私云协作的动态任务模型,支持在调度过程中灵活插入加密与验证子任务,提升跨云协作效率;同时引入细粒度隐私标签机制,动态适配多类隐私加密算法组,满足差异化隐私需求,提升隐私数据安全性。结合数据大小和隐私加密算法安全系数,提出混合云的安全性量化指标,并将混合云中隐私任务调度建模为完工时间、安全性及成本的三目标优化问题,为效率与安全的权衡提供依据。
    \item 设计了一种考虑卸载窗口优化的非支配排序遗传算法NSGA-OW,优化完工时间、安全性与成本三个目标。通过空闲窗口首次适应填充算法OW-FF提升混合云资源利用率,改进遗传算子加快收敛速度且改善调度质量,并改进精英保留策略加快调度算法速度,最终输出一组在完工时间、安全性与成本上最优的非支配解集,为用户提供多样的调度方案。
    \item 通过实验验证,本文的模型与算法表现出了显著的优势。与传统未考虑协作、细粒度隐私标签以及多目标优化的方法相比,本文方法在完工时间上提高了52.7\%,在隐私安全性上提升了31.9\%,同时保证了成本不劣于对比算法。此外,NSGA-OW算法在收敛速度与非支配解集质量上均优于传统多目标元启发式算法,进一步验证了其有效性。
\end{enumerate}

\section{论文结构安排}
本文对混合云中隐私任务调度问题进行研究,解决安全性与效率难以权衡的问题。在模型方面,还考虑数据所有者细粒度的隐私需求,提升了隐私数据安全性,并通过调度公私云的动态任务,并优化混合云空闲资源,提升了隐私任务处理效率。算法方面,提出了NSGA-OW多目标隐私任务调度算法,提高了收敛速度与非支配解集质量。论文的组织结构如图\ref{fig:thesis-outline}所示。本文主要划分为以下章节:

第\ref{chapter:intro}章,绪论。介绍混合云作为隐私数据处理平台的优势,并强调混合云隐私任务调度问题具有愈发重要的研究价值。再分析云计算任务调度与隐私任务调度的研究现状,根据隐私数据保护方法以及优化问题目标函数数量进行分类对比,指出现有研究存在平衡安全性与效率、实现细粒度隐私保护以及提升公私云协作效率三方面的不足。通过对比分析,明确本文研究方向。最后提出本文主要研究内容,并确定文章的组织结构。

第\ref{chapter:theories}章,介绍相关基础知识。包括混合云基本概念及其隐私加密算法,以及多目标优化理论。首先介绍混合云的基本概念与术语,以及混合云任务调度中常用的隐私加密算法。随后给出多目标优化问题的定义,并以NSGA-II算法为例,分析多目标优化算法在离散优化问题中的具体应用场景,为后续研究奠定理论基础。

第\ref{chapter:model}章,设计混合云动态细粒度隐私任务调度模型。通过公私云协作的动态任务模型,精确建模跨云协作中隐私加密算法的开销,为优化公私云协作效率打下基础。通过细粒度的隐私数据标签,对隐私数据进行精准保护,并提出混合云安全性指标,量化低安全系数隐私加密算法对隐私数据的潜在风险,从而提升混合云安全性,并为平衡安全性与效率提供基础。最后,将混合云中隐私任务调度问题建立为以完工时间、安全性和成本为目标的三目标优化问题。

第\ref{chapter:alg}章,提出多目标隐私任务调度算法。针对公私云跨云协作中的虚拟机空闲时段,设计卸载窗口首次适应填充算法OW-FF。并结合OW-FF算法,提出NSGA-OW算法,设计混合编码方案并改进遗传算子组合,还提出动态偏好精英选择策略,提升任务调度质量并加快收敛速度。

第\ref{chapter:experiements}章,通过实验验证本文模型与算法的有效性。从多目标优化性能指标、任务量与数据规模、数据隐私需求以及隐私加密算法计算开销四个方面分析调度算法。实验结果表明,所提模型与算法在收敛速度、解集质量及调度质量方面均优于对比方法。

第\ref{chapter:conclusion}章,总结和展望。对本文所提出的调度方法研究进行分析总结,说明现有研究中的不足之处,并对未来的研究方向进行预测。

\begin{figure}[htb!]
    \includesvg{img/文章结构.drawio.svg}
    \caption{论文组织结构图}\label{fig:thesis-outline}
\end{figure}

\section{本章小结}
本章首先介绍了混合云作为隐私数据处理平台的研究背景与意义。随后,梳理了当前研究现状,指出了现有研究存在难以平衡安全与效率,以及跨云调度效率低下和难以满足不同数据所有者细粒度的安全需求的问题。最后,明确了解决问题的方法,提出了研究内容,并确定了论文的组织结构框架。
