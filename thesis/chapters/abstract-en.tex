Hybrid cloud, by integrating the abundant resources of public cloud and the stringent data privacy protection of private cloud, has emerged as an ideal platform for processing private data. However, scheduling privacy tasks in hybrid cloud environments still faces challenges in balancing security and efficiency. Processing tasks solely in the private cloud ensures high security but suffers from resource limitations, while public-private cloud collaboration improves efficiency but introduces potential security risks. Existing studies rely on privacy constraints, such as restricting data processing to the private cloud or enforcing computationally intensive encryption algorithms during collaboration. However, such rigid constraints struggle to achieve equilibrium between security and efficiency: on one hand, data complying with privacy constraints may still face security risks during processing; on the other hand, fixed constraints under high hybrid cloud workloads leading to prolonged task completion times and reduced efficiency.

To address the challenge of balancing security and efficiency in hybrid cloud privacy task scheduling, this paper proposes a dynamic fine-grained privacy task scheduling model, which provides three quantitative metrics: makespan, security, and cost to guide the equilibrium between safety and efficiency. A multi-objective privacy task scheduling algorithm is further developed to optimize these metrics simultaneously, generating a set of non-dominated scheduling solutions that satisfy the trade-off between makespan (efficiency) and security. Users may select the optimal balance point from this Pareto-optimal solution set through manual intervention or automated multi-criteria decision-making methods.

The main contributions of this work include:

A dynamic fine-grained privacy task scheduling model leverages linear workflow technology to construct a dynamic task framework for public-private cloud collaboration. This model supports flexible insertion of encryption and verification subtasks during scheduling, enhancing hybrid cloud efficiency. Additionally, a fine-grained privacy labeling mechanism dynamically adapts multiple privacy encryption algorithm groups by mapping data owners’ privacy requirements to specific cryptographic combinations. By integrating data size and encryption security metrics, a quantifiable hybrid cloud security assessment framework is established. The privacy task scheduling problem is formulated as a three-objective optimization framework comprising makespan, security, and cost, thereby establishing a theoretical foundation for balancing the trade-offs between system efficiency and data security.

A novel non-dominated sorting genetic algorithm with offload window optimization (NSGA-OW) is designed to co-optimize makespan, security, and cost. For virtual machine idle time during cross-cloud collaboration, the Offload Window First-Fit Fill (OW-FF) algorithm dynamically detects and utilizes waiting intervals for task offloading to insert subsequent tasks, improving private cloud resource utilization. Enhanced with hybrid encoding and novel genetic operators VM block-based multi-point crossover and load-aware mutation, the NSGA-OW framework accelerates convergence and enhances scheduling quality. Prioritizing security and makespan optimization based on user preferences, the algorithm outputs a Pareto front of optimal solutions, allowing users to select security-efficiency trade-offs.

Experimental validation demonstrates the superiority of the proposed model and algorithms. Compared to traditional methods lacking public-private cloud collaboration, fine-grained privacy labeling, or multi-objective optimization, our approach achieves a 52.7\% reduction in makespan and a 31.9\% improvement in security metrics while maintaining cost competitiveness. Furthermore, NSGA-OW outperforms conventional multi-objective metaheuristic algorithms in convergence speed and Pareto front quality, further validating its effectiveness.

