混合云通过整合公有云丰富的资源和私有云严密的数据隐私保护,成为处理隐私数据的理想平台。然而,混合云中调度隐私任务存在安全性与效率难以平衡问题,由私有云独立处理虽安全但资源受限,而公私云协作处理虽高效但可能泄露隐私。已有研究依靠隐私约束保证安全性,例如限制数据只在私有云处理,或虽允许协作处理但只能使用高计算开销的隐私加密算法。然而,仅依靠隐私约束难以平衡安全与效率,一方面,即便满足跨云协作隐私约束条件的数据在实际处理中仍存在泄漏的风险;另一方面,在混合云高负载时,固定的约束导致资源利用率下降,进而使任务完工时间延长、效率下降。

针对混合云隐私任务调度中安全性与效率难以平衡的问题,本文构建了混合云中动态细粒度的隐私任务调度模型,为安全与效率的权衡提供完工时间、安全性与成本三大量化指标。再使用多目标隐私任务调度算法,同时优化这三个指标,并输出一组在完工时间(效率)与安全性上满足非支配条件的调度方案集合。可以通过用户手动干预或者自动的多标准决策等方法,从调度方案集合中选择安全与效率间的平衡点。

本文主要工作包括:

提出了一种动态细粒度的隐私任务调度模型。首先,完善了公私云协作的动态任务建模方法,使用线性工作流模型,将隐私加密算法看做调度过程中插入的加密与验证子任务,精准表示加密算法给隐私任务带来的额外开销,为调度算法优化提供了依据,从而提升公私云协作任务执行效率。其次,引入细粒度的隐私数据标签机制,能够为来自不同数据所有者的隐私数据选择合适的隐私加密算法,提高混合云的安全性。还基于数据大小与加密算法安全系数,提出了评估混合云系统安全性的量化指标。最后,将隐私任务调度问题建模为完工时间、安全性及成本的三目标优化问题,为平衡效率与安全提供理论依据。

设计了一种考虑卸载窗口优化的非支配排序遗传算法NSGA-OW,完成完工时间、安全性与成本的协同优化。提出了卸载窗口首次适应填充算法OW-FF,通过利用公私云协作时的资源空闲时段(即卸载窗口)处理其他任务,提高混合云资源利用率。针对混合云隐私任务调度问题特性,对多目标遗传算法进行改进,并结合OW-FF算法提出NSGA-OW算法。改进包括:设计适用于本文优化问题的混合编码方案;提出虚拟机分块多点交叉算子与负载感知变异算子,以加速算法收敛并提升调度质量;改进精英保留策略,优先优化完工时间与安全性目标,从而提高算法执行速度。最终输出一组在完工时间、安全性与成本上满足非支配条件的调度方案集合,以便用户选择安全性与效率平衡点。

通过实验验证本文提出的模型和算法的有效性。与传统未考虑动态公私云协作、细粒度隐私标签以及多目标优化的调度方法相比,本文在完工时间上降低了52.7\%,在安全性指标上提升了31.9\%,同时保证了成本不劣于对比算法。此外,NSGA-OW算法在收敛速度与调度方案解集质量上均优于传统多目标元启发式算法。

% (充分利用混合云资源,)
% 这以下的摘要缺少修正