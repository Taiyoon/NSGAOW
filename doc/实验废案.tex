隐私数据分布的研究:

本文设计两种任务与隐私数据的关系,一种为均匀分布模式,即所有任务对数据需要的概率相同,另一种为Zipf分布模式,用以模拟热点数据场景,即任务高频率需要少数几条数据,我们固定Zipf的参数为1.1。我们考虑到不同隐私任务可能依赖同一个数据以及,私有云包含1200条隐私数据,

% 原始数据(图片、150KB,加密70.33ms,解密102.16ms)
% 没有提供测试机数据,我们假设测试机的速度为3GHz a.k.a 3*10^9 Hz
% 因此加密开销为 175.825 cycle/bit((CPU周期/bit)), 解密开销为 255.4 cycle/bit((CPU周期/bit))
% 没有提供安全性量化方案,但是它提供身份验证、增强时间和数据完整性验证。适用于图片和文字数据加密
% 请注意,这哥们没考虑剔除加密算法的启动时间,所以数据量偏高,若实验发现加密开销不正确,可以酌情剔除启动时间

% 加密算法评估维度
% 评估指标    机密性服务(如AES)    完整性服务(如SHA)
% 密钥长度/哈希长度    直接影响暴力破解难度    决定抗碰撞攻击能力
% 算法结构    抗侧信道攻击能力评估    雪崩效应量化分析
% 标准化认证    NIST认证等级(FIPS 140-2等)    CRYPTREC/IEEE标准符合度
% 中国标签数据采用SM4国密算法,美国标签数据部署AES-256,欧洲标签数据 洲 于  ECRYPT密 码 算 法 征 集 活动 HC-128等算法。
% SM2+SM3\cite{ZhuYingJishitongxinxitongxiaguomiyuguojimimaxingnengbijiao2022} & 420.15 & 435.20 & 0.8 \\
% AES+SHA-1\cite{leiPrivacySecurityawareWorkflow2022} & 112.41 & 112.41 & 0.63 \\
% AES256+SM3 & 202.37 & 225.41  \\

% AES + SHA-256 & 158.22 & 162.30 & 0.95  \\
% ChaCha20-Poly1305 & 92.15 & 88.20 & 0.75  \\
% HC-256+SHA-1 & 195.75 & 183.20 & 0.68  \\
% AES+SHA-1\cite{leiPrivacySecurityawareWorkflow2022} & 112.41 & 112.41 & 0.63 \\
% \begin{table}[htbp]
%     目前找到的加密策略开销资料:
%     \begin{tblr}{rccc}
%         加密方案 & 加密开销 ((CPU周期/bit)) & 解密开销 ((CPU周期/bit)) & 安全系数  \\
%         SM4+SM3\cite{ZhuYingJishitongxinxitongxiaguomiyuguojimimaxingnengbijiao2022} & 599.53 & 605.63 & 1.0 \\
%         IDEA+TIGER\cite{leiPrivacySecurityawareWorkflow2022} & 235.47 & 235.47 & 1.0 \\
%         AES+ECC+SHA256\cite{williamAssessmentHybridCryptographic2022} & 175.825 & 255.400 & 0.2 \\
%         AES+SHA-1\cite{leiPrivacySecurityawareWorkflow2022} & 112.41 & 112.41 & 0.63 \\
%         RC4+MD5\cite{leiPrivacySecurityawareWorkflow2022} & 66.15 & 66.15 & 0.26 \\
%         HC-128+SHA-1\cite{sharifPerformanceAnalysisStream2010} & 195.75 & 183.20 & 0.68  \\
%     \end{tblr}
% \end{table}
% NSGA-OW的算法配置:在混合云隐私调度问题的编码研究中,针对问题的特殊性质,我们设计了交叉与变异算子:基于虚拟机分块的多点交叉算子和成本驱动变异算子,同时保留基础的两点交叉与位翻转算子,以保证解的多样性,并采用7:3的固定比例调用策略,以平衡算法性能。
% 算法参数:
% 虽然当前隐私安全模型尚未完全统一,但为便于比较,本文在统一的隐私安全模型下对比各调度算法的性能。
% 种群规模100,迭代次数=350,变异概率0.1, 交叉概率0.9
% 请注意,这三个算法不知道有没有坑
% \textbf{PSLS}~\cite{leiPrivacySecurityawareWorkflow2022}:
% 该算法先根据任务依赖计算排序值,确定每个任务的执行顺序;然后从公有云和私有云资源池中筛选符合隐私和安全标准的服务实例,并为任务选择在截止时间内费用最小或完成最快的最佳实例;最后按排序顺序调度任务并更新资源池,优化整体调度成本和执行时间。
% \textbf{MMA}~\cite{zhuTaskSchedulingMultiCloud2021}:该算法首先进行安全性检查剔除不符合安全要求的资源,再根据VM资源属性匹配为任务选出最佳初始分配,然后通过多轮迭代重分配任务,实现VM负载均衡,降低整体执行时间和成本。

% \subsection{数据分布特性对算法的影响}

% 在数据分布对算法的影响实验中,我们深入探讨了数据分布特性对算法性能的作用,重点关注热点数据与均匀数据分布对完工时间的影响,并进一步分析了数据集中存储于少数私有云虚拟机时的调度表现。实验设计了四种场景:场景1为热点数据(Zipf分布,参数=1.1)集中存储于少数私有云节点;场景2为热点数据分散存储于多数混合云节点;场景3为均匀数据分布集中存储于少数私有云节点;场景4为均匀数据分布分散存储于多数混合云节点,作为基线对照组。

% (可以尝试使用热力图等有趣的图表)

% (预期结果)
% 发现热点数据和集中存储分布会导致某些虚拟机负载过高,从而造成资源分配不均,显著延长完工时间并增加调度成本。
% \subsection{安全策略库规模对算法的影响}

% 实验设计两种配置以考察策略数量对调度算法的影响。配置1采用仅包含3条策略的较小策略集,具体包括安全等级为0.1(加解密开销60 CPU周期/bit)和0.9(600 CPU周期/bit)的协作策略,以及安全等级为1.0(无加解密开销)的私有云独占策略。配置2则包含10条策略,安全等级在0.1-0.9区间以0.1为步长均匀分布,加解密开销同样采用均匀间隔采样(60, 120, ..., 600 CPU周期/bit)。实验通过完工时间和安全性指标对调度性能进行评估,以全面分析安全策略库规模对算法的影响。

% (预期结果)
% 在固定任务到达数量和数据规模的条件下,比较不同安全策略库配置下的完工时间和成本指标,同时与其它多级安全设计算法进行对比。(预实验中不支持这个论断)
% 实验结果表明,增加策略数量可以使系统更精细地匹配数据特性,在保证安全性的前提下进一步降低调度成本和完工时间。

% \section{已知问题}

% 我们发现,实验中一条任务只对应着一条隐私数据,然而实际环境中,处理一个任务可能需要多条数据,还有可能数据来自混合云外部。但是没有依据,因此我们做了最简单假设,我们的任务只依赖储存在私有云中的一条隐私数。

% 不少隐私数据的加密方案难以获取准确的CPU开销,因为很难在一个统一的实验环境测试,同时不少加密算法可以使用硬件加速器,不使用CPU资源。因此本文对于加密算法的开销是简化的,假设他们都使用CPU资源,通过阅读论文,将它们具有处理时间的算法进行转化,将处理时间换算成CPU开销。

% 缺少实际数据

% 实验3 截止时间比较(与已有算法指标对比)

% 许多研究考虑调度具有截止时间的任务,为使任务满足截止时间需要使用较弱的安全策略,而本文的算法也支持调度具有截止时间的任务。本文任务下标,截止时间由紧迫到宽松进行排序,支持截止时间调度。

% 我们进行了实验,以确定在不同数据生成间隔下错过截止日期的任务百分比。如图XX所示。在XXX的任务量下,所提出的算法和原始 SPEA-II 的性能几乎相同,几乎所有任务都在指定的截止日期内响应,但通过缩短数据生成间隔,更多百分比的任务错过了截止日期。然而,与其他算法相比,所提出的算法具有更好的性能。由于所提出的算法试图最大限度地减少完工时间,因此这会导致错过截止日期的任务百分比降低。

% 在固定任务量条件下,通过调整数据生成间隔,记录各算法下错过截止日期任务的比例。如图~XX所示,在某固定任务负载下,我方算法与原始 SPEA-II 的截止时间满足率基本相当(几乎所有任务均在截止时间内完成);但当数据生成间隔缩短时,错过截止日期的任务比例明显上升。然而,相较于其它算法,我方算法在缩短数据生成间隔的场景下仍能保持较低的违约率,说明其在最小化完工时间方面具有优势,从而帮助降低任务未按截止日期完成的情况。
